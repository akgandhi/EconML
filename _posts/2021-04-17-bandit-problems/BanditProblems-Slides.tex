\documentclass{beamer}
%[handout,notes=show]
%\note{}



\usetheme{Montpellier}
%\usecolortheme{whale}



%\usepackage[utf8]{inputenc}
\usepackage{amsmath}
\usepackage{amsfonts}
\usepackage{amssymb}
\usepackage{mathrsfs}
\usepackage{bibentry} 
\usepackage{natbib}
%\usepackage[pdftex]{color,graphicx}



%\usepackage{hyperref}
\usepackage{verbatim}
\usepackage{natbib} 
\usepackage{setspace} 
\usepackage{listings}
\lstset{language=Matlab} 


\usepackage{mathptmx}
\usepackage[scaled=0.9]{helvet}
%\usepackage{courier}
\usepackage{soul}

\newtheorem{hyp}{Hypothesis}
\newtheorem{as}{Assumption}
\newtheorem{lem}{Lemma}
\newtheorem{defi}{Definition}
\newtheorem{prop}{Proposition}
\newtheorem{corr}{Corollary}
\newtheorem{theo}{Theorem}
\newtheorem{exl}{Example}
\newtheorem{h0}{H0}
\newtheorem{conj}{Conjecture}


\DeclareMathOperator*{\argmax}{argmax\;} 
\DeclareMathOperator*{\argmin}{argmin\;} 
\DeclareMathOperator{\sign}{sign}
\DeclareMathOperator{\Cov}{Cov}
\DeclareMathOperator{\Corr}{Corr}
\DeclareMathOperator{\Var}{Var}
\DeclareMathOperator{\supp}{supp}
\DeclareMathOperator{\tr}{tr}
\DeclareMathOperator{\diag}{diag}


\def\newblock{\hskip .11em plus .33em minus .07em}
 
 
 \newcommand{\bs}{\boldsymbol}
 
\makeatletter
\def\th@mystyle{%
    \normalfont % body font
    \setbeamercolor{block title example}{bg=green,fg=white}
    \setbeamercolor{block body example}{bg=green!20,fg=black}
    \def\inserttheoremblockenv{exampleblock}
  }
  
  
  
\makeatother
\theoremstyle{mystyle}
\newtheorem*{practice}{Practice problem} 
\newtheorem*{application}{Application} 
 
\beamertemplatenavigationsymbolsempty
\setbeamertemplate{footline}[frame number]
 
\author[Maximilian Kasy]{Maximilian Kasy}
\institute[Harvard]{Department of Economics, Harvard University}

\title[Bandits]{Econ 2148, spring 2019\\
Multi-armed bandits}
\date{}




\begin{document}



\nobibliography*



\frame{\titlepage }




\frame{
\frametitle{Agenda}
\begin{itemize}
\item Thus far: ``Supervised machine learning'' -- data are given.\\
Next: ``Active learning'' -- experimentation.
\item Setup: The multi-armed bandit problem.\\
Adaptive experiment with exploration / exploitation trade-off.
\item Two popular approximate algorithms:
\begin{enumerate}
\item Thompson sampling
\item Upper Confidence Bound algorithm
\end{enumerate}
\item Characterizing regret.
\item Characterizing an exact solution: Gittins Index.
\item Extension to settings with covariates (contextual bandits).
\end{itemize}
}

\frame{
\frametitle{Takeaways for this part of class}
\begin{itemize}
\item When experimental units arrive over time, and we can adapt our treatment choices, we can learn optimal treatment quickly.
\item Treatment choice: Trade-off between 
\begin{enumerate}
\item choosing good treatments now (exploitation), 
\item and learning for future treatment choices (exploration).
\end{enumerate}

\item Optimal solutions are hard, but good heuristics are available.
\item We will derive a bound on the regret of one heuristic.
\begin{itemize}
\item Bounding the number of times a sub-optimal treatment is chosen,
\item using large deviations bounds (cf. testing!).
\end{itemize}

\item We will also derive a characterization of the optimal solution in the infinite-horizon case. This relies on a separate index for each arm.
\end{itemize}

}

\section{The multi-armed bandit}



\frame{
\frametitle{The multi-armed bandit}
\framesubtitle{Setup}

\begin{itemize}
\item Treatments $D_t \in 1,\ldots, k$
\item Experimental units come in sequentially over time.\\
One unit per time period $t = 1, 2, \ldots$
\item Potential outcomes: i.i.d. over time, $Y_t = Y^{D_t}_t$,
\begin{align*}
Y^d_t & \sim F^d &
E[Y^d_t] &= \theta^d
\end{align*}
\item Treatment assignment can depend on past treatments and outcomes,
$$D_{t+1} = d_t(D_1, \ldots, D_{t}, Y_1, \ldots, Y_{t}).$$
\end{itemize}

}

\frame{
\frametitle{The multi-armed bandit}
\framesubtitle{Setup continued}

\begin{itemize}
\item Optimal treatment:
\begin{align*}
d^* &=\argmax_d \theta^d &
\theta^* &= \max_d \theta^d = \theta^{d^*}
\end{align*}
\item Expected regret for treatment $d$:
$$\Delta^d = E\left [Y^{d^*} - Y^d\right ] = \theta^{d^*} - \theta^d.$$

\item Finite horizon objective: Average outcome,
$$U_T= \tfrac{1}{T} \sum_{1 \leq t\leq T}  Y_t.$$
\item Infinite horizon objective: Discounted average outcome,
$$U_\infty=\sum_{t\geq 1} \beta^t Y_t$$

\end{itemize}

}


\frame{
\frametitle{The multi-armed bandit}
\framesubtitle{Expectations of objectives}

\begin{itemize}
\item Expected finite horizon objective:
$$E[U_T] = E\left [\tfrac{1}{T} \sum_{1 \leq t\leq T} \theta^{D_t}\right ]  $$

\item Expected infinite horizon objective:
$$E[U_\infty] = E\left [\sum_{t\geq 1} \beta^t \theta^{D_t}\right ]  $$

\item Expected finite horizon regret:\\
Compare to always assigning optimal treatment $d^*$.
$$R_T = E\left [\tfrac{1}{T} \sum_{1 \leq t\leq T} \left (Y_t^{d^*} - Y_t  \right )  \right ]=
 E\left [\tfrac{1}{T} \sum_{1 \leq t\leq T} \Delta^{D_t} \right ]$$



\end{itemize}


}


\frame{


\begin{practice}
\begin{itemize}
\item Show that these equalities hold.
\item Interpret these objectives.
\item Relate them to our decision theory terminology.
\end{itemize}
\end{practice}


}





\section{Two popular algorithms}

\frame{
\frametitle{Two popular algorithms}
\framesubtitle{Upper Confidence Bound (UCB) algorithm}

\begin{itemize}
\item Define
\begin{align*}
\bar Y^d_t &= \tfrac{1}{T^d_t} \sum_{1 \leq s\leq t} \bs 1(D_s = d) \cdot Y_s, \\
T^d_t &= \sum_{1 \leq s\leq t} \bs 1(D_s = d)\\
B^d_t &= B(T^d_t).
%(\psi^*)^{-1} \left (\frac{\alpha \log(t)}{T^d_t}   \right ).
\end{align*}
\item $B(\cdot)$ is a decreasing function, giving the width of the ``confidence interval.'' We will specify this function later.
\item At time $t+1$, choose
$$D_{t+1} = \argmax_d \bar Y^d_t + B^d_t.$$
\item ``Optimism in the face of uncertainty.''
\end{itemize}

}


\frame{
\frametitle{Two popular algorithms}
\framesubtitle{Thompson sampling}

\begin{itemize}
\item Start with a Bayesian prior for $\bs \theta$.
\item Assign each treatment with probability equal to the posterior probability that it is optimal.
\item Put differently, obtain one draw $\hat{\bs \theta}_{t+1}$ from the posterior given $(D_1, \ldots, D_{t}, Y_1, \ldots, Y_{t})$, and choose
$$D_{t+1} = \argmax_d \hat{\theta}^d_{t+1}.$$
\item Easily extendable to more complicated dynamic decision problems, complicated priors, etc.!
\end{itemize}

}


\frame{
\frametitle{Two popular algorithms}
\framesubtitle{Thompson sampling - the binomial case}

\begin{itemize}
\item Assume that $Y \in \{0,1\}$, $Y^d_t \sim Ber(\theta^d)$.
\item Start with a uniform prior for $\theta$ on $[0,1]^k$.
\item Then the posterior for $\theta^d$ at time $t+1$ is a $Beta$ distribution with parameters
\begin{align*}
\alpha^d_t &= 1 + T^d_t \cdot \bar Y^d_t,\\
\beta^d_t &=  1 + T^d_t \cdot (1-\bar Y^d_t).
\end{align*}

\item Thus
$$D_t = \argmax_d  \hat\theta_t.$$
where
$$\hat\theta_t^d \sim Beta(\alpha^d_t, \beta^d_t)$$
is a random draw from the posterior.
\end{itemize}

}


\section{Regret bounds}


\frame{
\frametitle{Regret bounds}

\begin{itemize}
\item Back to the general case.
\item  Recall expected finite horizon regret,
$$R_T = E\left [\tfrac{1}{T} \sum_{1 \leq t\leq T} \left (Y_t^{d^*} - Y_t  \right )  \right ]=
 E\left [\tfrac{1}{T} \sum_{1 \leq t\leq T} \Delta^{D_t} \right ].$$

\item Thus,
$$T\cdot R_T = \sum_d E[T^d_T] \cdot \Delta^d.$$ 

\item Good algorithms will have  $E[T^d_T]$ small when $\Delta^d >0$.

\item We will next derive upper bounds on  $E[T^d_T]$ for the UCB algorithm.

\item We will then state that for large $T$ similar upper bounds hold for Thompson sampling.
\item There is also a lower bound on regret across all possible algorithms which is the same, up to a constant.
\end{itemize}

}





\frame{
\frametitle{Probability theory preliminary}
\framesubtitle{Large deviations}

\begin{itemize}
\item Suppose that
$$ E[\exp(\lambda \cdot (Y-E[Y]))]  \leq \exp(\psi(\lambda)).$$
\item Let $\bar Y_T = \tfrac{1}{T} \sum_{1 \leq t\leq T} Y_t $ for i.i.d. $Y_t$.\\
Then, by Markov's inequality and independence across $t$,
\begin{align*}
P(\bar Y_T - E[Y] >\epsilon)
&\leq \frac{E[\exp(\lambda \cdot (\bar Y_T-E[Y]))]}{\exp(\lambda \cdot \epsilon)}\\
&= \frac{\prod_{1 \leq t\leq T}
E[\exp((\lambda/T) \cdot (Y_t-E[Y]))]}{\exp(\lambda \cdot \epsilon)}\\
&\leq \exp(T \psi(\lambda/T) - \lambda \cdot \epsilon).
\end{align*}


\end{itemize}


}

\frame{
\frametitle{Large deviations continued}

\begin{itemize}
\item Define the Legendre-transformation of $\psi$ as
$$\psi^*(\epsilon) = \sup_{\lambda \geq 0}\left [
\lambda \cdot \epsilon - \psi(\lambda)\right ].$$

\item Taking the $\inf$ over $\lambda$ in the previous slide implies
$$
P(\bar Y_T - E[Y] >\epsilon)\leq \exp(-T \cdot \psi^*(\epsilon)).
$$


\item For distributions bounded by $[0,1]$:\\
 $\psi(\lambda) = \lambda^2 / 8$ and $\psi^*(\epsilon) = 2 \epsilon^2$.
\item For normal distributions:\\
 $\psi(\lambda) = \lambda^2 \sigma^2 / 2$ and $\psi^*(\epsilon) = \epsilon^2 / (2 \sigma^2)$.


\end{itemize}

}



\frame{
\frametitle{Applied to the Bandit setting}

\begin{itemize}
\item Suppose that for all $d$
\begin{align*}
E[\exp(\lambda \cdot (Y^d-\theta^d))]  &\leq \exp(\psi(\lambda))\\
E[\exp(-\lambda \cdot (Y^d-\theta^d))]  &\leq \exp(\psi(\lambda)).
\end{align*}



\item Recall / define
\begin{align*}
\bar Y^d_t &= \tfrac{1}{T^d_t} \sum_{1 \leq s\leq t} \bs 1(D_s = d)\cdot Y_s, \\
B^d_t &= (\psi^*)^{-1} \left (\frac{\alpha \log(t)}{T^d_t}   \right ).
\end{align*}
 

\item Then we get
\begin{align*}
P(\bar Y^d_t - \theta^d > B^d_t )
&\leq \exp(- T^d_t \cdot \psi^*(B^d_t))\\
&= \exp(- \alpha \log(t)) = t^{-\alpha}\\
 P(\bar Y^d_t - \theta^d < - B^d_t )
&\leq  t^{-\alpha}.
\end{align*}
 

\end{itemize}

}


\frame{
\frametitle{Why this choice of $B(\cdot)$?}

\begin{itemize}
\item A smaller $B(\cdot)$ is better for exploitation.
\item A larger $B(\cdot)$ is better for exploration.
\item Special cases:
\begin{itemize}
\item Distributions bounded by $[0,1]$: 
$$B^d_t = \sqrt{ \frac{\alpha \log(t)}{2 T^d_t}}.$$
\item Normal distributions: 
$$B^d_t = \sqrt{ 2 \sigma^2 \frac{\alpha \log(t)}{ T^d_t} }.$$
\end{itemize}
\item The $\alpha \log(t)$ term ensures that coverage goes to $1$, but slow enough to not waste too much in terms of exploitation.
\end{itemize}

}

\frame{
\frametitle{When $d$ is chosen by the UCB algorithm}

\begin{itemize}
\item By definition of UCB, at least one of these three events has to hold when $d$ is chosen at time $t+1$:
\begin{align}
\bar Y^{d^*}_t + B^{d^*}_t &\leq \theta^*\\
\bar Y^d_t - B^d_t& > \theta^d\\
 B^d_t & > 2 \Delta^d.
\end{align}

\item 1 and 2 have low probability. By previous slide,
\begin{align*}
P\left (\bar Y^{d^*}_t + B^{d^*}_t \leq \theta^*\right ) &\leq  t^{-\alpha}, &
P\left (\bar Y^d_t - B^d_t > \theta^d\right ) &\leq  t^{-\alpha}.
\end{align*}

\item 3 only happens when $T^d_t$ is small. By definition of $B^d_t$, 3 happens iff
$$T^d_t < \frac{\alpha \log(t)}{\psi^* (\Delta^d / 2)}.$$

\end{itemize}

}

\frame{
\begin{practice}
Show that at least one of the statements 1, 2, or 3 has to be true whenever $D_{t+1} =d$, for the UCB algorithm.
\end{practice}
}


\frame{
\frametitle{Bounding $E[T^d_t]$}

\begin{itemize}
\item Let 
$$\tilde T^d_T = \left \lfloor \frac{\alpha \log(T)}{\psi^* (\Delta^d / 2)} \right \rfloor.$$
\item Forcing the algorithm to pick $d$ the first $\tilde T^d_T$ periods\\
 can only increase $T^d_T$.

\item  We can collect our results to get
\footnotesize
\begin{align*}
E[T^d_T] &= \sum_{1\leq t \leq T} \bs 1(D_t =d)
 \leq \tilde T^d_T  + \sum_{\tilde T^d_T < t \leq T} E[\bs 1(D_t =d)]\\
& \leq \tilde T^d_T  + \sum_{\tilde T^d_T < t \leq T} E[\bs 1(\text{(1) or (2) is true at } t)]\\
& \leq \tilde T^d_T  + \sum_{\tilde T^d_T < t \leq T} E[\bs 1(\text{(1)is true at } t)] + E[\bs 1(\text{(2) is true at }t)]\\
& \leq \tilde T^d_T  + \sum_{\tilde T^d_T < t \leq T} 2 t^{-\alpha+1}
 %\leq \tilde T^d_T  + 2 \sum_{1\leq t \leq T} t^{-\alpha}
 \leq \tilde T^d_T  + \frac{\alpha}{\alpha-2}.
\end{align*}
\end{itemize}

}

\frame{
\frametitle{Upper bound on expected regret for UCB}

\begin{itemize}
\item We thus get:

\begin{align*}
E[T^d_T] &\leq \frac{\alpha \log(T)}{\psi^* (\Delta^d / 2)} + \frac{\alpha}{\alpha-2},\\
R_T &\leq \tfrac{1}{T}\sum_d \left ( \frac{\alpha \log(T)}{\psi^* (\Delta^d / 2)} + \frac{\alpha}{\alpha-2}\right ) \cdot \Delta^d.
\end{align*}

\item Expected regret (difference to optimal policy) goes to $0$ at a rate of $O(\log(T) / T)$ -- pretty fast!

\item While the cost of ``getting treatment wrong'' is $\Delta^d$, the difficulty of figuring out the right treatment is of order $1/\psi^* (\Delta^d / 2)$.
Typically, this is of order $(1/\Delta^{d})^2$.
\end{itemize}

}


\frame{
\frametitle{Related bounds - rate optimality}

\begin{itemize}
\item \textbf{Lower bound}: 
Consider the Bandit problem with binary outcomes and any algorithm such that $E[T^d_t] = o(t^a)$ for all $a>0$.
Then
$$\liminf_{t\rightarrow \infty}  \tfrac{T}{\log(T)} \bar R_T \geq \sum_d \frac{\Delta^d}{kl(\theta^d, \theta^*)},$$
where $kl(p, q) = p \cdot\log(p/q) + (1-p) \cdot \log ((1-p)/(1-q))$.

\item \textbf{Upper bound for Thompson sampling}: In the binary outcome setting, Thompson sampling achieves this bound, i.e.,
$$\liminf_{t\rightarrow \infty}  \tfrac{T}{\log(T)} \bar R_T = \sum_d \frac{\Delta^d}{kl(\theta^d, \theta^*)}.$$

\end{itemize}

}

\section{Gittins index}


\frame{
\frametitle{Gittins index}
\framesubtitle{Setup}

\begin{itemize}
\item Consider now the discounted infinite-horizon objective,
$E[U_\infty] = E\left [\sum_{t\geq 1} \beta^t \theta^{D_t}\right ],  $
averaged over independent (!) priors across the components of $\theta$.
\item We will characterize the optimal strategy for maximizing this objective.
\item To do so consider the following, simpler decision problem:
\begin{itemize}
\item You can only assign treatment $d$.
\item You have to pay a charge of $\gamma^d$ each period in order to continue playing.
\item You may stop at any time, then the game ends.
\end{itemize}
\item Define $\gamma^d_t$ as the charge which would make you indifferent between playing or not, given the period $t$ posterior.
\end{itemize}

}


\frame{
\frametitle{Gittins index}
\framesubtitle{Formal definition}

\begin{itemize}
\item Denote by $\pi_t$ the posterior in period $t$, by $\tau(\cdot)$ an arbitrary stopping rule.
\item Define
\begin{align*}
\gamma^d_t &=
\sup \left \{ 
\gamma: \sup_{\tau(\cdot)} E_{\pi_t}\left [
\sum_{1\leq s \leq \tau} \beta^s \left (\theta^d - \gamma\right )
\right ] \geq 0
\right \}\\
&= \sup_{\tau(\cdot)} \frac{E_{\pi_t}\left [
\sum_{1\leq s \leq \tau} \beta^s \theta^d\right ]}{E_{\pi_t}\left [
\sum_{1\leq s \leq \tau} \beta^s \right ]}
\end{align*}

\item Gittins and Jones (1974) prove: The optimal policy in the bandit problem always chooses
$$D_t = \argmax_d \gamma^d_t.$$
\end{itemize}

}



\frame{
\frametitle{Heuristic proof (sketch)}

\begin{itemize}
\item Imagine a per-period charge for each treatment is set initially  equal to $\gamma^d_1$.
\begin{itemize}
\item Start playing the arm with the highest charge, continue until it is optimal to stop.
\item At that point, the charge is reduced to $\gamma^d_t$.
\item Repeat. 
\end{itemize}

\item This is the optimal policy, since:
\begin{enumerate}
\item It maximizes the amount of charges paid.
\item Total expected benefits are equal to total expected charges.
\item There is no other policy that would achieve expected benefits bigger than expected charges.
\end{enumerate}
\end{itemize}

}

\section{Contextual bandits}

\frame{
\frametitle{Contextual bandits}

\begin{itemize}
\item A more general bandit problem:
\begin{itemize}
\item For each unit (period), we observe covariates $X_t$.
\item Treatment may condition on $X_t$.
\item Outcomes are drawn from a distribution $F^{x,d}$, with mean $\theta^{x,d}$.
\end{itemize}
\item In this setting Gittins' theorem fails when the prior distribution of $\theta^{x,d}$ is not independent across $x$ or across $d$.
\item But Thompson sampling is easily generalized.\\
For instance to a hierarchical Bayes model:
\begin{align*}
Y^d| X=x, \bs \theta, \bs \alpha, \bs \beta &\sim Ber(\theta^{x,d})\\
\theta^{x,d} | \bs \alpha, \bs \beta &\sim Beta(\alpha^d, \beta^d)\\
(\alpha^d, \beta^d) &\sim \pi.
\end{align*}
\item This model updates the prior for $\theta^{x,d}$ not only based on observations with $D=d, X=x$, but also based on observations with $D=d$ but different values for $X$.

\end{itemize}

}


\section{References}

\frame{
\frametitle{References}
\begin{itemize}
\item  

\begin{verse}
\bibentry{RegretBandit2012}.
\end{verse} 


\item  

\begin{verse}
\bibentry{Thompson2018}.
\end{verse} 

\item 
\begin{verse}
\bibentry{weber1992gittins}.
\end{verse} 

\end{itemize}

}



\bibliographystyle{apalike}
\nobibliography{../../../../../research/library}




\end{document}





 